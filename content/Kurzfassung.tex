%% This is file Kurzfassung.tex of ZwickTeX-Template
\begingroup
\onehalfspacing
%% END of Header
%~~~~~~~~~~~~~~~~~~~~~~~~~~~~~~~~~~~~~~~~~~~~~~~~~~~~~~~~~~~~~~~~~~~~~~~~~~~~~~~~~



% Kurzfassung Deutsch
\addchap{Kurzfassung}

%==============================================
FH Kufstein \\
\getStudiengang \\
Kurzfassung der \getThesis\xspace \getTitel  \\
\getAutor \\
\getBetreuer \\
Vorname Zweitbetreuername \\
%==============================================


Die Kurzfassung dient dem Aufbau einer internen Datenbank und der Weitergabe
von Informationen an Interessenten außerhalb der FH Kufstein.
Die Kurzfassung ist sowohl auf Deutsch als auch auf Englisch zu erstellen (2 Versionen).

Bei der englischsprachigen Kurzfassung muss auch der Titel der Diplomarbeit
in englischer Sprache angegeben werden. Die Kurzfassung sollte jeweils nicht mehr
als 1 Seite umfassen und im Fließtext mit möglichst wenigen Aufzählungen und ohne
Abbildungen, Fußnoten oder Formatierungen gestaltet sein. Sie soll in knapper Form \ldots

die Problemstellung, \\
die Zielsetzung, \\
die Methodik \\
und die wichtigsten Ergebnisse darstellen. 

Die Kurzfassung ist nach dem Abkürzungsverzeichnis und vor dem
eigentlichen Text in die Arbeit einzubinden sowie auf Datenträgern oder via E-Mail
mit den gebundenen Exemplaren der Diplomarbeit abzugeben. Die Kurzfassung soll
folgende Angaben enthalten:

1. Zeile, linker oberer Rand: FH Kufstein \\
2. Zeile: „[Studiengang]“ \\
3. Zeile: Kurzfassung der Diplomarbeit ... (Titel einfügen) \\
4. Zeile: Name des Studierenden \\
5. Zeile: Erstbetreuer \\ 
6. Zeile: Zweitbetreuer \\

Text der Kurzfassung – Fließtext, ca. 350 Wörter









%~~~~~~~~~~~~~~~~~~~~~~~~~~~~~~~~~~~~~~~~~~~~~~~~~~~~~~~~~~~~~~~~~~~~~~~~~~~~~~~~~
% Zusammenfassung Englisch
\addchap{Abstract}

%==============================================
FH Kufstein \\
\getStudiengang \\
Abstract of \getThesis\xspace Englischer Titel  \\
\getAutor \\
\getBetreuer \\
Vorname Zweitbetreuername \\
%==============================================


Same procedure as before\ldots









%~~~~~~~~~~~~~~~~~~~~~~~~~~~~~~~~~~~~~~~~~~~~~~~~~~~~~~~~~~~~~~~~~~~~~~~~~~~~~~~~~
%% BEGIN Footer
\endgroup
%% End of Footer
%% EOF Kurzfassung.tex

